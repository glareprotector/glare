\documentclass[a4paper,10pt]{article}

\usepackage[margin=1in]{geometry}
\usepackage{enumitem}
\usepackage[pdftex]{graphicx}
\usepackage{bbm}
\usepackage{amsfonts}
\usepackage{amsthm}
\usepackage{subfigure}

\theoremstyle{definition}
\newtheorem{defn}{Definition}
\newtheorem{obs}{Observation}
\begin{document}

\section{Parameterization a Beta Distribution by its Mode and Shape Parameter}
Here I briefly jot down the new parameterization we will use for the Beta distribution.  Before, we had been parameterization a beta distribution by its mean and a dispersion parameter related to its variance.  We modelled the mean using a linear model.  However, the drawback was that when the variance of a beta distribution is large, its mean and mode are not close to each other.  What we really wanted was to model the mode of a Beta distribution using a linear model.  With this parameterization found from literature \cite{newbeta}, we can now do that.

\subsection{Beta Distribution Facts}
\begin{itemize}
\item $p(x) \sim x^{\alpha-1}(1-x)^{\beta-1}$
\item mode = $\frac{\alpha-1}{\alpha+\beta-2}$
\item unimodal if both $\alpha>1$ and $\beta>1$
\end{itemize}

\subsection{New Parameterization}
Let $m$ be the desired mode of the beta distribution, and $s>-1$ be a shape parameter.  They show a new parameterization $Beta'(m,s)$ which is equivalent to a (old parameterization) $Beta(\alpha,\beta)$ distribution where 
\begin{eqnarray}
\alpha &=& 1 + sm \\
\beta &=& 1 + s(1-m)
\end{eqnarray}

First, we can check that the mode of a $Beta'(m,s)$ distribution is indeed $m$.  The mode of this distribution is:
\begin{eqnarray}
\frac{\alpha-1}{\alpha+\beta-2} = \frac{sm}{sm + s(1-m)} = m
\end{eqnarray}

\subsection{Unimodality}
They also claim that a $Beta'(m,s)$ distribution is unimodal iff $s>0$.  This is true because $m$ is between 0 and 1, and plugging $s>0$ into equation 1 and 2 always leads to $\alpha>1$ and $\beta>1$.


\begin{thebibliography}{10}
\bibitem{newbeta} Dorp, J. On Some Elicitation Procedures for Distributions with Bounded Support with Applications in Pert.  Page 2. 
\end{thebibliography}
\end{document}